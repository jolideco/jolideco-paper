% Define document class
\documentclass[twocolumn]{aastex631}

% Filler text
\usepackage{blindtext}
\usepackage{amsmath}
\usepackage[utf8]{inputenc}
\usepackage{showyourwork}
\usepackage{amssymb}
cd ..
\DeclareMathOperator*{\argmax}{arg\,max}
\newcommand{\chandra}{\textit{Chandra}~}
\newcommand{\xmm}{\textit{XMM}~}
\newcommand{\fermi}{\textit{Fermi}-LAT~}
\newcommand{\jolideco}{\textit{Jolideco}~}
\newcommand{\todo}[1]{\textcolor{red}{TODO: #1}\PackageWarning{TODO:}{#1!}}

\newcommand{\vlk}[1]{{\color{blue} [VLK: #1]}}

% Begin!
\begin{document}

% Title
    \title{\jolideco: Joint Likelihood Deconvolution of Astronomical Images in the Presence of Poisson Noise}

% Author list
    \author[0000-0003-4568-7005]{Axel Donath}
    \author[0000-0002-0905-7375]{Aneta Siemiginowska}
    \author[0000-0002-3869-7996]{Vinay Kashyap}
    \author[0000-0000-0000-0000]{David van Dyk}


% Abstract with filler text
    \begin{abstract}
        We present a new method for (Jo)int (li)kelihood  (Deco)nvolution (\jolideco) of a set of astronomical observations of the same sky region in the presence of Poisson noise.
        The method reconstructs an image from a set of observations
        by optimizing the a-posteriori joint Poisson likelihood of all
        observations under an patch based image prior. The patch
        prior is parameterised by a standard Gaussian Mixture model (GMM)
        learned from astronomical images at other wavelenghts and
        adapted to the structures expected to see in the image.
        By applying the method to simulated data we show that
        the combination of mutiple observations leads to an
        improved reconstruction quality in the regime with S/N ratio of.
        We show also that the method yields superior reconstruction quality
        to alternative standard methods such as the Richardson-Lucy method.
    \end{abstract}

    % Main body with filler text


    \section{Introduction}
    The quality of all astronomical images is affected by the limited angular resolution of the instrument
    or telescope. In addition the images are affected by the presence of noise and non-uniform exposure.

    In literature there have been multiple efforts to 
    
    Any astronomical observation is affected by the limited angular resolution of the telescope.

    \begin{itemize}
        \item Richardson \& Lucy (RL)\cite{Richardson1972} \cite{Lucy1974}
        \item Joint likelihood RL: \cite{Ingaramo2014} has found that RL take the best out of each image and create a "merged" image
        \item  Problem of RL decomposing structures into point sources: \cite{Reeves1994} \cite{Fish1995}
        \item Multi-scale LIRA prior of course: \cite{Esch2004, Connors2011}, advantage of error estimates / Bayesian sampling. However prior is rather "ad-hoc", no physical information / assumptions. Just "smoothness"
        \item What abput d3po and d4po? Priors based on physical assumptions, decomposition into point and diffuse flux, but in 3d/4d \cite{Selig2015}\cite{Pumpe2018}
    \end{itemize}
    
    
    % \begin{figure*}
    %     \script{rl_decomposition.py}
    %     \begin{centering}
    %         \includegraphics[width=\textwidth]{figures/richardson-lucy-decomposition.pdf}
    %         \caption{
    %             Decomposition of the RL algorithm for various number of iterations. \todo{Improve figure}
    %         }
    %         \label{fig:rl_decomposition}
    %     \end{centering}
    % \end{figure*}

    \subsection{A note on Deep Learning Methods}
    \cite{Li2014} have introduced a baseline architecture for a deep convolutional neural network for image deblurring.
    The convolutional nature of the network does not make it suitable for the task. As convolution introduces smoothing.
    However the network can learn an "inverse kernel", which needs to be larger because of the uncertainty principle (\todo{reference...}).
    The large kernels are introduced by separability assumption...
    In the limit of Poisson noise it is desirable to trace the full a-postiori likelihood, which is not possibly for most network architectures. 

    \section{Method}
    \subsection{Poisson Joint Likelihood}
    Our goal is to recover an image $\mathbf{x}$ from a set of multiple
    low counts observations.
    Most generally we assume our total dataset consists of $M$ individual
    observations of the same region of the sky, which are jointly
    modelled. The assumption is that the underlying \textit{true} emission image,
    we are looking for does not change with time. Under this assumption these
    datasets can be for example:

    \begin{itemize}
        \item Different observations of one instrument or telescope at different times and observation conditions. E.g., multiple observations of \chandra with different offset angles and exposure times.
        \item Observations of different telescopes, which operate in the same wavelength range. E.g., a \chandra
        and \xmm observation of the same region in the sky.
        \item A single observation of one telescope with different data quality categories and different associated instrument
        response functions. E.g., event classes for \fermi.
    \end{itemize}

    Or a even an arbitrary combination of the possibilities listed above.

    For each individual observation $m$ the predicted counts can be modelled by forward
    folding the unknown flux image $\mathbf{x}$ with the individual (per observation) instrument response:

    \begin{equation}
        \label{eq:model}
        \mathbf{\lambda}_m = \mathrm{PSF}_m \circledast \left(\mathbf{E}_m \cdot (\mathbf{x} + \mathbf{B}_m) \right)
    \end{equation}

    Where the expected counts $\lambda_i$ are given by the convolution of the true underlying
    flux distribution $\mathbf{x}$ with the $\mathrm{PSF}$. Additionally
    the observation specific exposure $\mathbf{E}$ and background emission $\mathbf{B}$ can be
    taken into account.

    Given a single observation $m$ of a true underlying flux distribution
    $\mathbf{x}$ and assuming the noise in each pixel $i$ in the recorded counts image
    $\mathbf{d}$ follows a Poisson distribution, the total likelihood $\mathcal{L}_m$
    of obtaining the measured image from a model image of the expected
    counts $\lambda_i$ with $N$ pixels is given by:

    \begin{equation}
        \label{eq:poisson}
        \mathcal{L}_m\left( \mathbf{d} | \mathbf{\lambda} \right) = \prod_i^N \frac{{e^{ - d_i } \lambda_i ^ {d_i}}}{{d_i!}}
    \end{equation}


    By taking the logarithm and dropping the constant terms one can transform the
    product into a sum over pixels, which is also often called the \textit{Cash}
    \citep{Cash1979} fit statistics:

    \begin{equation}
        \label{eq:cash}
        \mathcal{C}\left( \mathbf{d} | \mathbf{\lambda} \right) = \sum_i^N \lambda_i - d_i \log{\lambda_i}
    \end{equation}


    Bayes rule:

    \begin{equation}
        \label{eq:bayes}
        P(\theta|\textbf{D}) = P(\theta ) \frac{\mathcal{C}(\textbf{D} |\theta)}{P(\textbf{D})}
    \end{equation}

    The total objective function $\mathcal{L}$ is given by:
    \begin{equation}
        \label{eq:total}
        \mathcal{L}\left( \mathbf{d_m} | \mathbf{x} \right) = \sum_m^M \mathcal{C}\left( \mathbf{d_m} | \mathbf{x} \right) - \beta \cdot \mathcal{P}(\mathbf{x})
    \end{equation}

    Where $\mathcal{C}\left( \mathbf{d_m} | \mathbf{x} \right)$ represents the summed log-likelihood
    for an individual observation $m$. Additionally the function includes a prior term $\mathcal{P}(\mathbf{x})$
    and a factor $\beta$ to adjust the weight of the prior with respect to the likelihood term. \todo{To be clarified: the weight
    does not really make it a prior, more a regularisation term...}

    \subsection{Priors}
    Just include patch priors, or LIRA as well, or uniform?

    \subsubsection{Patch-Prior}
    To improve the reconstruction quality of images for inverse problems,
    such as denoising, inpainting or deconvolution it is useful to have prior
    assumptions on the statistics of the image to guide the optimization based
    image reconstruction process. However images are high
    dimensional and thus difficult to capture the global image statistics
    taking into account all pixel to pixel correlations. 

    By recognising that on smal scale images often contain basic structure such as 
    edges, corners, periodic patterns or region of constant brightness. 
    The idea of patch priors was first introduced by \cite{Zoran2011}.
    Their main idea was to learn the statistics of natural images on a small
    spatial scales instead. For this they proposed to split images from a
    representative dataset into small patches of size 8x8 pixels.

    Then they learned a 64 dimensional Gaussian Mixture Model (GMM) on the
    extracted patches, treating each pixel as an independent dimension in
    the model, with $k=200$ components. They showed that the GMM prior led to much improved
    image reconstructions, compared to comparable approaches (which ones?).

    One of the main advantages of the GMM is the possibility to
    evaluate its log-likelihood in closed form. It is given by:

    \begin{equation}
        \ell_{GMM}(\theta) = \sum_{i=1}^n \log \left( \sum_{k=1}^K \pi_k N(x_i;\mu_k, \sigma_k^2) \right )
    \end{equation}

    With parameters $\theta = \{\mu_1,\ldots,\mu_K,\sigma_1,\ldots,\sigma_K,\pi_1,\ldots,\pi_K\}$.
    To use the learned GMM as prior for the reconstruction of another image, the image is split into
    overlapping patches. For each of the patches the log-likelihood for each of the GMM components is
    evaluated. The resulting grid of overlapping patches is illustrated in Figure~\ref{fig:patches}.

    For each 8x8 pixel patch the likelihood of all components of the GMM is evaluated. Then 
    \begin{equation}
        \hat{k} = \argmax{\ell_{GMM}(\theta)}
    \end{equation}


    \begin{equation}
        \mathcal{P}(x) = \sum_n \log{p_{\hat{k}}(\mathbf{P}_n x)}
    \end{equation}

    $\mathbf{P}_n$ is a matrix that extracts the $n$-th patch
    from the image $\mathbf{x}$ to be reconstructed.
    $p(\mathbf{P}_n x)$ is the probability of that patch
    under the GMM.


    \begin{figure}[ht!]
        \script{patches.py}
        \begin{centering}
            \includegraphics[width=\linewidth]{figures/patches.pdf}
            \caption{
                Grid of overlapping patches of size 8x8 pixels. The overlap size was chosen to be 2 pixels.
            }
            \label{fig:patches}
        \end{centering}
    \end{figure}

    \cite{Bouman2016} later adapted the patch prior reconstruction to be used
    with radio astronomy data.
    They have shown that the reconstructed image only weakly depends on the choice
    of the reference data on which the the patch prior is learned. They found
    equivalent results for GMMs learned on natural images and specifically
    simulated images of black hole ring structures.

    \subsubsection{Image Normalisation}
    Astromomical images typically show a much higher dynamic range compared to 
    natural images. This is mostly due to the existence of point sources, which rarely occur
    in natural images, but a very common in astronomical images because of objects 
    at hig distances.
    The GMM patch prior is learned on normalized images, where the intensity values
    are constrained between 0 and 1. To evaluate For the likelihood term the image
    intensity need to be conserved.


    The choice of the image normalisation allows to adjust the contrast of the image
    and e.g. enhance low intensity structures, which also enhances the regularising
    effect of the patch prior.

    \todo{Which normalisation to choose, atan, asinh, log, inverse-cdf?}

    \subsubsection{Cycle Spinning}
    To avoid artifact due to the choice of the grid of overlapping patches we propose
    three variations:

    \begin{itemize}
        \item Cycle spining by randomly shifting the image by a given number of pixels in x and y direction
        \item Sub pixel cycle spinning, by randomly distributing the bightness of a given pixel to a 4 pixel grid
        \item A randomly chosen patch grid, such that each pixel is at least covered once.
        The random grid of patches was proposed by \cite{Parameswaran2018}.
    \end{itemize}

    \subsection{Cross Validation}
    The use of multiple dataset allows to sue cross-validation to prevent overfitting.
    One of the standard techniques in "machine learning"...
    

    \subsection{Calibration of Predicted Counts}
    Each dataset comes with systematic errors on absolute position and background normalisation
    
    \begin{itemize}
        \item allow for systematic shifts between observations 
        \item allow for background scaling per observation $\alpha_m$
    \end{itemize}

    \begin{equation}
        \label{eq:model-npred-calibration}
        \mathbf{\lambda}_m = \mathrm{PSF}_m \circledast \left(\mathbf{E}_m \cdot (\phi_m(\mathbf{x}| \delta_x, \delta_y) + \alpha_m \cdot \mathbf{B}_m) \right)
    \end{equation}


    
    Show example residual images before and after calibration...


    \section{Implementation}
    \subsection{Jolideco Framework}
    The goal of the reconstruction process is to optimize the a-posteriori
    likelihood defined by Equation~\ref{eq:total}. Given that each pixel
    in the reconstructed image represents an independent parameter
    this represents a high dimensional optimization problem.
    Differentiable programming frameworks such as \texttt{PyTorch}
    allow for solving these kind of high dimensional modeling problems, by using
    back-propagation and adapted optimization methods such as stochastic gradient
    decent.

    The \jolideco method was implemented as an independent Python package, 
    based on \texttt{PyTorch} as optimization back-end. We tried to define a modular,
    object oriented code structure, to allow parts of the algorithm to be
    flexible and interchangeable, such as choice of image normalisation scales,
    the choice of GMM models, optimization methods and serilisation formats for the 
    reconstruction results as well as corresponding diagnostic information
    such as the trace of the posterior and prior likelihood values.

    The package is available at \url{https://github.com/jolideco/jolideco}

    \begin{itemize}
        \item direct maximum a posteriori optimization using Pytorch etc.
        \item auto gradient computation
        \item ML optimizers such as ADAM
        \item GPU support
    \end{itemize}

    In addition we use \texttt{Scikit-Learn}~\citep{Skimage2014} to learn the GMM.

    We use \texttt{Astropy}~\citep{Astropy2018} for FITS serialisation and
    handling WCS transforms.

    We use \texttt{Numpy}~\citep{Numpy2020} for handling data arrays and
    \texttt{Matplotlib}~\citep{Hunter2007} for plotting.

    For optimization and internal data handling we use \texttt{PyTorch}~\citep{Pytorch2019}.

    \subsection{Jolideco GMM Library}
    We provide a selection of learned GMMs to be used with the patch prior for reproducibility and download. 
    Convenience for future users. \url{https://github.com/jolideco/jolideco-gmm-library}
    
    \subsubsection{Zoran \& Weiss}
    As baseline reference GMM we use the original model provided by \cite{Zoran2011}. The model was learned
    from the Berkley image database, by splitting the images into 8x8 patches. 

    \subsubsection{GLEAM data}
    High signal to noise Radio data from \cite{HurleyWalker2022}

    \subsubsection{NRAO Jets, Jets, Jets}
    For the specific application of jet analysis we learned a GMM patch prior from jet images. It encodes the prior knowledge of the preferred direction of the jet. 

    \begin{itemize}
        \item One with horizontal direction, where the image to be analysed should be rotated. E.g. if prior knowledge from radio data is available.
        \item One with randomised direction, if now prior knowledge is available.
    \end{itemize}
    

    \subsection{Computational Performance}
    We conduct as as series of performance benchmarks (see Appendix?) to asses the scalability of the method to a large number of observations
    as well as large images. Results are available in \url{https://github.com/jolideco/jolideco-benchmarks}
    

    
    \section{Experiments}
    \subsection{Test Datasets}
    \todo{Vinay writes this?}
    \vlk{yes}

    \vlk{We have devised four distinct arrangements of combinations of point and extended sources of differents shapes to test the algorithms.  The arrangements are 
    \begin{itemize}
        \item[ ``Gauss":] A set of four point sources arranged around an extended Gaussian source
        \item[``Points":] Point sources arranged with different separations
        \item[``Shield":] Disk shaped flat extended source with superposed point sources and linear jet-like features
        \item[``Spiral":] Extended thin double-spiral with a flat disk at the center and point sources adjacent to it
    \end{itemize}
    The arrangements are shown in Figure~TBD, and the details of the locations and brightness of each component is listed in Table~TBD.
    }

    We use the test datasets provided by Vinay et al.
    First we evaluate the performance of the method on a set of simulated observations.
    For this we assume:

    \begin{itemize}
        \item An instrument with good angular resolution, but low effective area (e.g. like Chandra)
        \item An instrument with worse angularr resolution, bur higher effective area (e.g. like XMM)
    \end{itemize}

    For both scenarios we assume a Gaussian PSF of sizes $\sigma = 2$ pixels and  $\sigma = 5$.
    As true flux we consider the following scenarios:

    \begin{itemize}
        \item Point source with varying distances and brightness
        \item Disk and point sources of varying brightness
        \item Spiral and point sources of varying brightness
    \end{itemize}

    Background levels of $\lambda_{Bkg}= 0.001, 0.01 \textrm{and} 0.1 \textrm{counts/pixel}$. 
    
    Show results of experiments on simulated toy datasets

    

    \subsection{Results}

    The package is available at \url{https://github.com/jolideco/jolideco-comparison}


    \section{Application Examples}
    Show results from real observations

    \subsection{Deep \chandra Observation}

    \subsection{Combined \xmm and \chandra Observation}

    \subsection{\fermi Event Classes}

    \section{Reproducibility}

    The paper is available at \url{https://github.com/jolideco/jolideco-paper}
    


    \section{Summary \& Conclusions}
    In this work we presented a new method for image deconvolution and denoising in the presence of Poisoon noise.
    Jolideco is great...

    Extend \jolideco to handl spectral dimension at the same time.

    \section*{Acknowledgements}
    This work was conducted under the auspices of the CHASC International Astrostatistics Center.
    CHASC is supported by NSF grants DMS-21-13615, DMS-21-13397, and DMS-21-13605; by the UK Engineering
    and Physical Sciences Research Council [EP/W015080/1]; and by NASA 18-APRA18-0019.
    We thank CHASC members for many helpful discussions, especially Xiao-Li Meng and Katy McKeough.
    DvD was also supported in part by a Marie-Skodowska-Curie RISE Grant (H2020-MSCA-RISE-2019-873089)
    provided by the European Commission.
    Aneta Siemiginowska, Vinay Kashyap, and Doug Burke further acknowledge support from NASA
    contract to the Chandra X-ray Center NAS8-03060.

    \newpage
    \bibliography{bib.bib}
\end{document}
